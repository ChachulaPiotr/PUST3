\chapter {Pomiar w punkcie pracy}
\label{zad1l}

\section{Komunikacja z obiektem}

Komunikacja z obiektem odobywa się za pomocą funkcji napisanych w środowisku MatLab. Najprostszy program użyty w tym projekcie, który posłużył także do późniejszego pobierania odpowiedzi skokowych znajduje się poniżej (jest to fragment skryptu \verb|L3_1.m|).


%\begin{lstlisting}[style=custommatlab,frame=single]
\begin{lstlisting}[style=custommatlab]


    addpath('F:\SerialCommunication'); % add a path to the functions
    initSerialControl COM3 % initialise com port
   
    N = 420;
    step_response = zeros(N+1,1);
    i = 0;
    
    while(i<=N)
        %% obtaining measurements
        measurements = readMeasurements(1:7); % read measurements from 1 to 1
        
        %% processing of the measurements and new control values calculation
        disp([measurements(1),i]);
        step_response(i+1)=measurements(1);
        
        %% sending new values of control signals
        sendNonlinearControls(29)  % new corresponding control valuesdisp(measurements); % process measurements
      
        i=i+1;
        step_response(i)=measurements(1);
        
        waitForNewIteration(); % wait for new batch of measurements to be ready
        
    end

\end{lstlisting} 

W tym zadaniu używamy funkcji \verb|sendNonlinearControls|, który wysyła sterowanie w sposób symulujący nieliniowość obiektu. Napisany skrypt działał poprawnie, pozwala na sterowanie sygnałami G1, W1 oraz pomiar T1.

\section{Punkt pracy}

Doprowadzono obiekt do punktu pracy, tj. ustawiono wartości sygnałów W1 na 50, G1 na 29 i poczekano na ustabilizowanie obiektu (ponieważ obiekt jest rzeczywisty wahania temperatury są nieuniknione, zwłaszcza biorąc pod uwagę fakt lokalizacji stanowiska nr 4 w miejscu obok którego przechodzi dużo osób - wszelkie pomiary teraźniejsze oraz późniejsze mogą być zaburzone właśnie przez to). Wartość temperatury w punkcie pracy wynosi T1=\num{35,43}\degree C.
